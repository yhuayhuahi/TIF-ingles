\section{INTRODUCCIÓN}

    En el campo de la Ingenieria de Sistemas, poseer un vocabulario técnico amplio y preciso en inglés es fundamental para el desarrollo profesional y académico. El inglés técnico no solo facilita la comprensión de literatura especializada, sino que también mejora la capacidad de comunicación en un entorno globalizado, donde el inglés es el idioma predominante de la ciencia y la tecnología. \newline
    
    El objetivo de esta investigación es compilar un compendio exhaustivo de términos clave en inglés, organizados alfabéticamente, que son relevantes para la Ingeniería de Sistemas. Cada letra del abecedario será explorada en profundidad para identificar la mayor cantidad de palabras posibles que se utilizan en nuestra disciplina. \newline
    
    A través de esta investigación, buscamos no solo enriquecer nuestro vocabulario técnico, sino también proporcionar una herramienta útil para estudiantes y profesionales que deseen mejorar su dominio del inglés en contextos técnicos. Al final de este trabajo, esperamos haber creado un recurso valioso que contribuya al desarrollo de competencias lingüísticas.

\section{OBJETIVOS}

    \subsection{Objetivo General}
        
        Compilar un compendio exhaustivo de términos clave en inglés, organizados alfabéticamente, que sean relevantes para la Ingeniería de Sistemas, con el fin de enriquecer el vocabulario técnico de estudiantes y profesionales en esta disciplina.

    \subsection{Objetivos Especificos}

        \begin{itemize}
            \item Identificar y recolectar palabras técnicas en inglés relevantes para la Ingeniería de Sistemas que comiencen con cada letra del abecedario. 
            \item Clasificar y organizar las palabras recolectadas en categorías temáticas dentro de la Ingeniería de Sistemas, tales como desarrollo de software, redes, bases de datos, y ciberseguridad. 
            \item Proporcionar ejemplos del uso de cada palabra encontrada.
            \item Desarrollar un recurso didáctico que pueda ser utilizado en cursos de inglés técnico para Ingeniería de Sistemas, mejorando así las competencias lingüísticas de los estudiantes. Evaluar la relevancia y aplicabilidad de los términos técnicos recopilados a través de encuestas y entrevistas con profesionales y académicos del área.
        \end{itemize}

\section{METODOLOGÍA}

    Para la realización de esta investigación, se llevará a cabo un proceso de recolección y clasificación de términos técnicos en inglés relevantes para la Ingeniería de Sistemas. Se realizará una búsqueda exhaustiva de términos técnicos en inglés relacionados con la Ingeniería de Sistemas. Se utilizarán fuentes confiables como libros de texto, artículos académicos, sitios web especializados y glosarios técnicos. Los términos recolectados serán organizados alfabéticamente, comenzando con la letra A y avanzando hasta la letra Z.

\section{CUERPO DE LA INVESTIGACIÓN}
    % Subseccion letra G
    \subsection{Letra G}
    \begin{itemize}
        % Gateway
        \item \textbf{Gateway}
        \begin{itemize}
            \item The gateway connects different networks.
            \item Every network needs a gateway for communication.
        \end{itemize}
        % Gigabyte
        \item \textbf{Gigabyte}
        \begin{itemize}
            \item The hard drive has a capacity of 500 gigabytes.
            \item A gigabyte is equal to 1,024 megabytes.
        \end{itemize}
        % Graphical User Interface
        \item \textbf{Graphical User Interface}
        \begin{itemize}
            \item The software has a user-friendly graphical user interface.
            \item A graphical user interface makes interaction easier.
        \end{itemize}
        % GigaHertz
        \item \textbf{Gigahertz}
        \begin{itemize}
            \item The processor speed is measured in gigahertz.
            \item Higher gigahertz means a faster CPU.
        \end{itemize}
    \end{itemize}

    % Subseccion letra H
    \subsection{Letra H}
    \begin{itemize}
        % Hard Drive
        \item \textbf{Hard Drive}
        \begin{itemize}
            \item The hard drive stores all the data.
            \item A larger hard drive can hold more files.
        \end{itemize}
        % Hyperlink
        \item \textbf{Hyperlink}
        \begin{itemize}
            \item Click the hyperlink to visit the website.
            \item Hyperlinks connect different web pages.
        \end{itemize}
        % HTML
        \item \textbf{HTML}
        \begin{itemize}
            \item HTML is the standard markup language for web pages.
            \item Learning HTML is essential for web development.
        \end{itemize}
        % Host
        \item \textbf{Host}
        \begin{itemize}
            \item The server acts as a host for the website.
            \item Each device on the network is a host.
        \end{itemize}
    \end{itemize}

    % Subseccion letra I
    \subsection{Letra I}
    \begin{itemize}
        % Internet
        \item \textbf{Internet}
        \begin{itemize}
            \item The internet connects millions of computers worldwide.
            \item You can find information on the internet.
        \end{itemize}
        % IP Address
        \item \textbf{IP Address}
        \begin{itemize}
            \item Every device has a unique IP address.
            \item An IP address identifies a device on the network.
        \end{itemize}
        % Input Device
        \item \textbf{Input Device}
        \begin{itemize}
            \item A keyboard is an input device.
            \item Input devices allow users to interact with the computer.
        \end{itemize}
        % Integrated Circuit
        \item \textbf{Integrated Circuit}
        \begin{itemize}
            \item An integrated circuit is found in every computer.
            \item Integrated circuits are used in various electronic devices.
        \end{itemize}
    \end{itemize}

    % Subseccion letra J
    \subsection{Letra J}
    \begin{itemize}
        % Java
        \item \textbf{Java}
        \begin{itemize}
            \item Java is a popular programming language.
            \item Many applications are developed using Java.
        \end{itemize}
        % JSON
        \item \textbf{JSON}
        \begin{itemize}
            \item JSON is used for data interchange.
            \item The API returns data in JSON format.
        \end{itemize}
        % JUnit
        \item \textbf{JUnit}
        \begin{itemize}
            \item JUnit is used for testing Java applications.
            \item Writing tests in JUnit improves code quality.
        \end{itemize}
        % JDBC
        \item \textbf{JDBC}
        \begin{itemize}
            \item JDBC is used to connect Java applications to databases.
            \item Using JDBC simplifies database interactions.
        \end{itemize}
    \end{itemize}

    % Subseccion letra K
    \subsection{Letra K}
    \begin{itemize}
        % Kernel
        \item \textbf{Kernel}
        \begin{itemize}
            \item The kernel is the core of the operating system.
            \item A stable kernel is essential for system performance.
        \end{itemize}
        % Keyword
        \item \textbf{Keyword}
        \begin{itemize}
            \item In programming, keywords have special meanings.
            \item Keywords cannot be used as variable names.
        \end{itemize}
        % Kilobyte
        \item \textbf{Kilobyte}
        \begin{itemize}
            \item A kilobyte is equal to 1,024 bytes.
            \item Small files are often measured in kilobytes.
        \end{itemize}
        % Kubernetes
        \item \textbf{Kubernetes}
        \begin{itemize}
            \item Kubernetes is used for container orchestration.
            \item Managing applications with Kubernetes is efficient.
        \end{itemize}
    \end{itemize}

    % Subseccion letra L
    \subsection{Letra L}
    \begin{itemize}
        % Linux
        \item \textbf{Linux}
        \begin{itemize}
            \item Linux is an open-source operating system.
            \item Many servers run on Linux.
        \end{itemize}
        % Loop
        \item \textbf{Loop}
        \begin{itemize}
            \item A loop repeats a block of code.
            \item Loops are essential in programming.
        \end{itemize}
        % Library
        \item \textbf{Library}
        \begin{itemize}
            \item A library provides reusable code.
            \item Developers use libraries to save time.
        \end{itemize}
        % Load Balancer
        \item \textbf{Load Balancer}
        \begin{itemize}
            \item A load balancer distributes traffic across servers.
            \item Using a load balancer improves system reliability.
        \end{itemize}
    \end{itemize}
    %Subsecion letra M
    \subsection{Letra M}
    \begin{itemize}
        % Microprocessor
        \item \textbf{Microprocessor}
        \begin{itemize}
            \item The computer has a microprocessor.
            \item A microprocessor is a samll chip.
        \end{itemize}
        % Mainframe
        \item \textbf{Mainframe}
        \begin{itemize}
            \item The company uses a mainframe for data storage.
            \item A mainframe is very powerful.
        \end{itemize}
        % Modem
        \item \textbf{Modem}
        \begin{itemize}
            \item I need a modem to connect to the internet.
            \item The modem is small and white.
        \end{itemize}
        % Motherboard
        \item \textbf{Motherboard}
        \begin{itemize}
            \item The motherboard is inside the computer.
            \item Every computer has a motherboard.
        \end{itemize}
        % Malware
        \item \textbf{Malware}
        \begin{itemize}
            \item Malware can harm your computer.
            \item Good antivirus software can stop malware.
        \end{itemize}
    \end{itemize}
    % Subseccion letra N
    \subsection{Letra N}
    \begin{itemize}
        % Network
        \item \textbf{Network}
        \begin{itemize}
            \item The office has a computer network.
            \item A network connects many computers.
        \end{itemize}
        % Node
        \item \textbf{Node}
        \begin{itemize}
            \item Each computer is a node in the network.
            \item The node is part of a larger system.
        \end{itemize}
        % Nanotechnology
        \item \textbf{Nanotechnology}
        \begin{itemize}
            \item Nanotechnology is used in medicine.
            \item Nanotechnology works on a very small scale.
        \end{itemize}
        % Notebook (Computer)
        \item \textbf{Notebook (Computer)}
        \begin{itemize}
            \item He uses a notebook for his work.
            \item A notebook is a portable computer.
        \end{itemize}
    \end{itemize}
    % Subseccion letra O
    \subsection{Letra O}
    \begin{itemize}
        % Operating System
        \item \textbf{Operating System}
        \begin{itemize}
            \item Windows is an operating system.
            \item Every computer needs an operating system.
        \end{itemize}
        % optical Fiber
        \item \textbf{Optical Fiber}
        \begin{itemize}
            \item Optical fiber cables are very fast.
            \item The internet uses optical fiber.
        \end{itemize}
        % Open Source
        \item \textbf{Open Source}
        \begin{itemize}
            \item Linux is an open source software.
            \item Open source software is free to use.
        \end{itemize}
        % Overclocking
        \item \textbf{Overclocking}
        \begin{itemize}
            \item Overclocking makes the computer faster.
            \item Overclocking can heat up the CPU.
        \end{itemize}
    \end{itemize}
    % Subsecion letra P
    \subsection{Letra P}
    \begin{itemize}
        % Programming
        \item \textbf{Programming}
        \begin{itemize}
            \item Yourdyy is learning programming.
            \item Programming is writing computer code.
        \end{itemize}
        % Processor
        \item \textbf{Processor}
        \begin{itemize}
            \item The processor is the brain of the computer.
            \item A fast processor makes a computer quick.
        \end{itemize}
        % Peripheral
        \item \textbf{Peripheral}
        \begin{itemize}
            \item A mouse is a peripheral device.
            \item Printers and scanners are peripherals.
        \end{itemize}
        % Protocol
        \item \textbf{Protocol}
        \begin{itemize}
            \item HTTP is a web protocol.
            \item Protocols help computers communicate.
        \end{itemize}
    \end{itemize}
    % Subseccion letra Q
    \subsection{Letra Q}
    \begin{itemize}
        % Query
        \item \textbf{Query}
        \begin{itemize}
            \item Joao made a query in the database.
            \item A query finds information quickly.
        \end{itemize}
        % Quantum Computing
        \item \textbf{Quantum Computing}
        \begin{itemize}
            \item Quantum computing is very advanced.
            \item Quantum computers can solve hard problems.
        \end{itemize}
        % Queue
        \item \textbf{Queue}
        \begin{itemize}
            \item The printer has a print queue.
            \item A queue organizes tasks in order.
        \end{itemize}
    \end{itemize}
    % Subseccion letra R
    \subsection{Letra R}
    \begin{itemize}
        % Router
        \item \textbf{Router}
        \begin{itemize}
            \item The router connects to the internet.
            \item Every home network needs a router..
        \end{itemize}
        % RAM (Random Access Memory)
        \item \textbf{RAM (Random Access Memory)}
        \begin{itemize}
            \item RAM makes the computer run faster.
            \item Adding more RAM helps with multitasking.
        \end{itemize}
        % Repository
        \item \textbf{Repository}
        \begin{itemize}
            \item The code is stored in a repository.
            \item A repository helps with version control.
        \end{itemize}
        % Runtime
        \item \textbf{Runtime}
        \begin{itemize}
            \item The program needs a runtime environment.
            \item Runtime errors happen during execution.
        \end{itemize}
    \end{itemize}
    % Subsection letra S
    \subsection{Letra S}
    \begin{itemize}
        % Server
        \item \textbf{Server}
        \begin{itemize}
            \item Our company upgraded to a more powerful server to handle increased traffic.
            \item The server crashed during peak hours, causing significant downtime for our users.
        \end{itemize}
        % Scripting
        \item \textbf{Scripting}
        \begin{itemize}
            \item Scripting languages like Python and JavaScript are essential for automating tasks.
            \item He used a scripting tool to extract data from multiple websites efficiently.
        \end{itemize}
        % Security
        \item \textbf{Security}
        \begin{itemize}
            \item Implementing robust security measures can prevent unauthorized access to sensitive data. 
            \item The IT department conducted a security audit to identify potential vulnerabilities.
        \end{itemize}
        % Software
        \item \textbf{Software}
        \begin{itemize}
            \item The software development team released a new version of the application with enhanced features. 
            \item It is important to keep your software updated to protect against security threats.
        \end{itemize}
        % Storage
        \item \textbf{Storage}
        \begin{itemize}
            \item Cloud storage solutions provide scalable options for data backup and retrieval. 
            \item He ran out of storage space on his laptop and had to delete some files.
        \end{itemize}
        % Synchronization
        \item \textbf{Synchronization}
        \begin{itemize}
            \item Data synchronization ensures that all devices have the most up-to-date information. 
            \item The calendar app allows synchronization across multiple devices.
        \end{itemize}
        % Scalability
        \item \textbf{Scalability}
        \begin{itemize}
            \item Scalability is a critical factor in the design of distributed systems. 
            \item The application was built with scalability in mind, allowing it to handle more users over time.
        \end{itemize}
    \end{itemize}
    % Subsection letra T
    \subsection{Letra T}
    \begin{itemize}
        % Tag
        \item \textbf{Tag}
        \begin{itemize}
            \item HTML tags are used to define the structure and content of web pages.
            \item The programmer added metadata tags to improve the website's search engine optimization.
        \end{itemize}
        % Taskbar
        \item \textbf{Taskbar}
        \begin{itemize}
            \item The taskbar in Windows provides quick access to frequently used applications.
            \item He customized his taskbar to include shortcuts for his favorite tools and programs.
        \end{itemize}
        % Terabyte
        \item \textbf{Terabyte}
        \begin{itemize}
            \item Modern data centers often have storage capacities measured in terabytes.
            \item A single terabyte can hold approximately 250,000 high-quality photos.
        \end{itemize}
        % Telemetry
        \item \textbf{Telemetry}
        \begin{itemize}
            \item The spacecraft sent telemetry data back to Earth to monitor its systems and performance.
            \item Telemetry is crucial in IoT devices to track and report sensor data in real-time.
        \end{itemize}
        % Terminal
        \item \textbf{Terminal}
        \begin{itemize}
            \item The developer used the terminal to execute commands and run scripts on the server.
            \item Understanding terminal commands is essential for effective system administration.
        \end{itemize}
        % Thread
        \item \textbf{Thread}
        \begin{itemize}
            \item Multithreading allows a program to perform multiple tasks simultaneously.
            \item The software engineer optimized the code to reduce thread contention and improve performance.
        \end{itemize}
        % Throughput
        \item \textbf{Throughput}
        \begin{itemize}
            \item Network throughput measures the amount of data successfully transmitted over a network in a given time.
            \item Increasing the server's throughput improved the website's loading times for users.
        \end{itemize}
        % Thumbnail
        \item \textbf{Thumbnail}
        \begin{itemize}
            \item Thumbnails provide a quick preview of images in a gallery.
            \item She clicked on the thumbnail to view the full-sized photo.
        \end{itemize}
        % Token
        \item \textbf{Token}
        \begin{itemize}
            \item The authentication system uses tokens to verify user identities without storing passwords.
            \item Each API request includes a token to ensure secure communication between the client and server.
        \end{itemize}
        % Toolbar
        \item \textbf{Toolbar}
        \begin{itemize}
            \item The toolbar at the top of the software provides shortcuts to commonly used functions.
            \item He customized his toolbar to include only the tools he uses frequently.
        \end{itemize}
        % Traceroute
        \item \textbf{Traceroute}
        \begin{itemize}
            \item The network engineer used traceroute to diagnose where the data packets were being delayed.
            \item Traceroute is a valuable tool for mapping the path data takes through a network.
        \end{itemize}
        % Tuple
        \item \textbf{Tuple}
        \begin{itemize}
            \item A tuple is a finite ordered list of elements often used in database queries.
            \item In Python, tuples are immutable, meaning their contents cannot be changed after creation.
        \end{itemize}
    \end{itemize}
    % Subsection letra U
    \subsection{Letra U}
    \begin{itemize}
        % UDP
        \item \textbf{UDP}
        \begin{itemize}
            \item UDP (User Datagram Protocol) is used for time-sensitive transmissions such as video playback or online gaming.
            \item Unlike TCP, UDP does not guarantee the delivery of data packets.
        \end{itemize}
        % UI
        \item \textbf{UI}
        \begin{itemize}
            \item A well-designed UI (User Interface) enhances the user experience significantly.
            \item The development team focused on improving the UI to make the application more user-friendly.
        \end{itemize}
        % UEFI
        \item \textbf{UEFI}
        \begin{itemize}
            \item UEFI (Unified Extensible Firmware Interface) is a modern version of BIOS with more features and a better user interface.
            \item Many new computers use UEFI instead of the traditional BIOS.
        \end{itemize}
        % UML
        \item \textbf{UML}
        \begin{itemize}
            \item UML (Unified Modeling Language) is used to create diagrams that represent the structure and behavior of a system.
            \item The software engineers used UML to design the architecture of the new system.
        \end{itemize}
        % Unicode
        \item \textbf{Unicode}
        \begin{itemize}
            \item Unicode provides a unique number for every character, no matter the platform, program, or language.
            \item The latest version of Unicode includes over 143,000 characters.
        \end{itemize}
        % URI
        \item \textbf{URI}
        \begin{itemize}
            \item A URI (Uniform Resource Identifier) is a string of characters used to identify a resource on the internet.
            \item The web page's URI includes both the URL and additional information like the anchor.
        \end{itemize}
        % URL
        \item \textbf{URL}
        \begin{itemize}
            \item A URL (Uniform Resource Locator) is the address used to access a web resource.
            \item She bookmarked the URL of her favorite website for easy access.
        \end{itemize}
        % USB
        \item \textbf{USB}
        \begin{itemize}
            \item USB (Universal Serial Bus) is the most common interface for connecting peripheral devices to a computer.
            \item He plugged the USB drive into his computer to transfer files.
        \end{itemize}
        % Uptime
        \item \textbf{Uptime}
        \begin{itemize}
            \item The server's uptime is a crucial metric for maintaining reliable services.
            \item They monitor the system's uptime to ensure minimal downtime for users.
        \end{itemize}
    \end{itemize}
    % Subsection letra V
    \subsection{Letra V}
    \begin{itemize}
        % Variable
        \item \textbf{Variable}
        \begin{itemize}
            \item The value of the variable can change during the execution of the program.
            \item In statistical analysis, a variable is any characteristic that can take on different values.
        \end{itemize}
        % Vector
        \item \textbf{Vector}
        \begin{itemize}
            \item The graphics software uses vector images to ensure scalability without loss of quality.
            \item In physics, a vector is a quantity that has both magnitude and direction.
        \end{itemize}
        % Virtualization
        \item \textbf{Virtualization}
        \begin{itemize}
            \item Virtualization technology enables the creation of virtual instances of hardware resources.
            \item By using virtualization, businesses can optimize resource utilization and reduce costs.
        \end{itemize}
        % Virus
        \item \textbf{Virus}
        \begin{itemize}
            \item A computer virus can replicate itself and spread from one computer to another.
            \item Installing antivirus software can help protect your system from malicious viruses.
        \end{itemize}
        % VPN
        \item \textbf{VPN}
        \begin{itemize}
            \item A VPN, or Virtual Private Network, provides a secure connection over the internet.
            \item Using a VPN can help protect your privacy and sensitive data when browsing online.
        \end{itemize}
    \end{itemize}
    % Subsection letra W
    \subsection{Letra W}
    \begin{itemize}
        % Web Hosting
        \item \textbf{Web Hosting}
        \begin{itemize}
            \item They chose a reliable web hosting provider to ensure their website had minimal downtime.
            \item Web hosting services are essential for making your website accessible on the internet.
        \end{itemize}
        % Wearable Technology
        \item \textbf{Wearable Technology}
        \begin{itemize}
            \item Wearable technology includes devices like smartwatches and fitness trackers.
            \item Advances in wearable technology have made it possible to monitor health metrics in real-time.
        \end{itemize}
        % Web Service
        \item \textbf{Web Service}
        \begin{itemize}
            \item A web service allows different applications to communicate with each other over the internet.
            \item They developed a web service to integrate their platform with third-party applications.
        \end{itemize}
        % Wireless
        \item \textbf{Wireless}
        \begin{itemize}
            \item Wireless technology enables communication without the need for physical connections.
            \item The office upgraded to a wireless network to provide more flexibility for employees.
        \end{itemize}
    \end{itemize}
    % Subsection letra X
    \subsection{Letra X}
    \begin{itemize}
        % XML
        \item \textbf{XML}
        \begin{itemize}
            \item XML, or Extensible Markup Language, is used to store and transport data.
            \item Many web services use XML for communication between different systems.
        \end{itemize}
        % XSS
        \item \textbf{XSS}
        \begin{itemize}
            \item XSS, or Cross-Site Scripting, is a web security vulnerability that allows the injection of malicious scripts.
            \item Protection against XSS attacks includes input validation and escaping user input.
        \end{itemize}
        % XPath
        \item \textbf{XPath}
        \begin{itemize}
            \item XPath is a language used to navigate through elements and attributes in XML documents.
            \item Developers use XPath to extract specific data from XML documents efficiently.
        \end{itemize}
        % Xcode
        \item \textbf{Xcode}
        \begin{itemize}
            \item Xcode is Apple's Integrated Development Environment (IDE) for creating applications for iOS and macOS.
            \item With Xcode, developers can write, compile, and debug applications in various programming languages.
        \end{itemize}
        % XAML
        \item \textbf{XAML}
        \begin{itemize}
            \item XAML, or Extensible Application Markup Language, is used to design user interfaces in Windows applications.
            \item Developers of WPF and UWP applications use XAML to define the appearance and behavior of user interface elements.
        \end{itemize}
    \end{itemize}
    % Subsection Letra Y
    \subsection{Letra Y}
    \begin{itemize}
        % YAML
        \item \textbf{YAML}
        \begin{itemize}
            \item YAML, or YAML Ain't Markup Language, is a human-readable data serialization standard.
            \item YAML is often used for configuration files and data exchange between languages with different data structures.
        \end{itemize}
        % YARN
        \item \textbf{YARN}
        \begin{itemize}
            \item YARN, or Yet Another Resource Negotiator, is a resource management and job scheduling technology in the Apache Hadoop ecosystem.
            \item YARN enhances the scalability and efficiency of processing large datasets by allowing multiple data processing engines.
        \end{itemize}
        % Yacc
        \item \textbf{Yacc}
        \begin{itemize}
            \item Yacc, or Yet Another Compiler Compiler, is a tool used in Unix systems to generate parsers for interpreting structured text.
            \item Developers use Yacc to create syntax analyzers for compilers, which convert high-level code into machine code.
        \end{itemize}
        % YouTrack
        \item \textbf{YouTrack}
        \begin{itemize}
            \item YouTrack is a project management and issue tracking tool developed by JetBrains.
            \item Teams use YouTrack to track bugs, manage projects, and collaborate efficiently.
        \end{itemize}
        % YUI
        \item \textbf{YUI}
        \begin{itemize}
            \item YUI, or Yahoo User Interface Library, is an open-source JavaScript and CSS library for building interactive web applications.
            \item Although no longer actively maintained, YUI provided developers with tools and utilities for creating rich web experiences.
        \end{itemize}
    \end{itemize}
    % Subsection letra Z
    \subsection{Letra Z}
    \begin{itemize}
        % Zero-day
        \item \textbf{Zero-day}
        \begin{itemize}
            \item A zero-day vulnerability is a software flaw that is unknown to the software vendor and for which no patch is available.
            \item Zero-day exploits can be particularly dangerous as they can be used by attackers before the vendor has a chance to address the vulnerability.
        \end{itemize}
        % ZFS
        \item \textbf{ZFS}
        \begin{itemize}
            \item ZFS, or Zettabyte File System, is a high-performance file system designed for data integrity and scalability.
            \item ZFS includes features like data compression, snapshots, and dynamic striping for improved data management.
        \end{itemize}
        % ZooKeeper
        \item \textbf{ZooKeeper}
        \begin{itemize}
            \item Apache ZooKeeper is a distributed coordination service for managing large clusters of servers.
            \item ZooKeeper helps maintain configuration information, naming, and synchronization services across distributed applications.
        \end{itemize}
        % Zig
        \item \textbf{Zig}
        \begin{itemize}
            \item Zig is a general-purpose programming language designed for robustness, optimality, and maintainability.
            \item Developers use Zig for system programming, offering features like manual memory management and minimal runtime.
        \end{itemize}
        % Z-shell
        \item \textbf{Z-shell}
        \begin{itemize}
            \item Z-shell, or Zsh, is a powerful Unix shell that extends the Bourne shell with numerous features.
            \item Zsh includes features like advanced autocompletion, globbing, and a rich scripting language for enhanced productivity.
        \end{itemize}
    \end{itemize}

\section{CONCLUSIONES}

\section{RECOMENDACIONES}

\clearpage
\begin{thebibliography}{X}
    \bibitem{Autor1} Complete reference here
    \bibitem{Autor2} Complete reference here
    \bibitem{Autor3} Complete reference here
\end{thebibliography}
