\section{INTRODUCCIÓN}

    En el campo de la Ingenieria de Sistemas, poseer un vocabulario técnico amplio y preciso en inglés es fundamental para el desarrollo profesional y académico. El inglés técnico no solo facilita la comprensión de literatura especializada, sino que también mejora la capacidad de comunicación en un entorno globalizado, donde el inglés es el idioma predominante de la ciencia y la tecnología.
    
    El objetivo de esta investigación es compilar un compendio exhaustivo de términos clave en inglés, organizados alfabéticamente, que son relevantes para la Ingeniería de Sistemas. Cada letra del abecedario será explorada en profundidad para identificar la mayor cantidad de palabras posibles que se utilizan en nuestra disciplina.
    
    A través de esta investigación, buscamos no solo enriquecer nuestro vocabulario técnico, sino también proporcionar una herramienta útil para estudiantes y profesionales que deseen mejorar su dominio del inglés en contextos técnicos. Al final de este trabajo, esperamos haber creado un recurso valioso que contribuya al desarrollo de competencias lingüísticas.

\section{OBJETIVOS}

    \subsection{Objetivo General}
        
        Compilar un compendio exhaustivo de términos clave en inglés, organizados alfabéticamente, que sean relevantes para la Ingeniería de Sistemas, con el fin de enriquecer el vocabulario técnico de estudiantes y profesionales en esta disciplina.

    \subsection{Objetivos Especificos}

        \begin{itemize}
            \item Identificar y recolectar palabras técnicas en inglés relevantes para la Ingeniería de Sistemas que comiencen con cada letra del abecedario. 
            \item Clasificar y organizar las palabras recolectadas en categorías temáticas dentro de la Ingeniería de Sistemas, tales como desarrollo de software, redes, bases de datos, y ciberseguridad. 
            \item Proporcionar ejemplos del uso de cada palabra encontrada.
            \item Elaborar un glosario técnico bilingüe (inglés-español) que facilite la comprensión y el aprendizaje de los términos identificados.
            \item Desarrollar un recurso didáctico que pueda ser utilizado en cursos de inglés técnico para Ingeniería de Sistemas, mejorando así las competencias lingüísticas de los estudiantes. Evaluar la relevancia y aplicabilidad de los términos técnicos recopilados a través de encuestas y entrevistas con profesionales y académicos del área.
        \end{itemize}

\section{METODOLOGÍA}


\section{CUERPO DE LA INVESTIGACIÓN}
\lipsum[13-15]
    %Subsecion letra M
    \subsection{Letter M}
    \begin{itemize}
        % Microprocessor
        \item \textbf{Microprocessor}
        \begin{itemize}
            \item The computer has a microprocessor.
            \item A microprocessor is a samll chip.
        \end{itemize}
        % Mainframe
        \item \textbf{Mainframe}
        \begin{itemize}
            \item The company uses a mainframe for data storage.
            \item A mainframe is very powerful.
        \end{itemize}
        % Modem
        \item \textbf{Modem}
        \begin{itemize}
            \item I need a modem to connect to the internet.
            \item The modem is small and white.
        \end{itemize}
        % Motherboard
        \item \textbf{Motherboard}
        \begin{itemize}
            \item The motherboard is inside the computer.
            \item Every computer has a motherboard.
        \end{itemize}
        % Malware
        \item \textbf{Malware}
        \begin{itemize}
            \item Malware can harm your computer.
            \item Good antivirus software can stop malware.
        \end{itemize}
    \end{itemize}
    % Subseccion letra N
    \subsection{Letter N}
    \begin{itemize}
        % Network
        \item \textbf{Network}
        \begin{itemize}
            \item The office has a computer network.
            \item A network connects many computers.
        \end{itemize}
        % Node
        \item \textbf{Node}
        \begin{itemize}
            \item Each computer is a node in the network.
            \item The node is part of a larger system.
        \end{itemize}
        % Nanotechnology
        \item \textbf{Nanotechnology}
        \begin{itemize}
            \item Nanotechnology is used in medicine.
            \item Nanotechnology works on a very small scale.
        \end{itemize}
        % Notebook (Computer)
        \item \textbf{Notebook (Computer)}
        \begin{itemize}
            \item He uses a notebook for his work.
            \item A notebook is a portable computer.
        \end{itemize}
    \end{itemize}
    % Subseccion letra O
    \subsection{Letra O}
    \begin{itemize}
        % Operating System
        \item \textbf{Operating System}
        \begin{itemize}
            \item Windows is an operating system.
            \item Every computer needs an operating system.
        \end{itemize}
        % optical Fiber
        \item \textbf{Optical Fiber}
        \begin{itemize}
            \item Optical fiber cables are very fast.
            \item The internet uses optical fiber.
        \end{itemize}
        % Open Source
        \item \textbf{Open Source}
        \begin{itemize}
            \item Linux is an open source software.
            \item Open source software is free to use.
        \end{itemize}
        % Overclocking
        \item \textbf{Overclocking}
        \begin{itemize}
            \item Overclocking makes the computer faster.
            \item Overclocking can heat up the CPU.
        \end{itemize}
    \end{itemize}
    % Subsecion letra P
    \subsection{Letra P}
    \begin{itemize}
        % Programming
        \item \textbf{Programming}
        \begin{itemize}
            \item Yourdyy is learning programming.
            \item Programming is writing computer code.
        \end{itemize}
        % Processor
        \item \textbf{Processor}
        \begin{itemize}
            \item The processor is the brain of the computer.
            \item A fast processor makes a computer quick.
        \end{itemize}
        % Peripheral
        \item \textbf{Peripheral}
        \begin{itemize}
            \item A mouse is a peripheral device.
            \item Printers and scanners are peripherals.
        \end{itemize}
        % Protocol
        \item \textbf{Protocol}
        \begin{itemize}
            \item HTTP is a web protocol.
            \item Protocols help computers communicate.
        \end{itemize}
    \end{itemize}
    % Subseccion letra Q
    \subsection{Letra Q}
    \begin{itemize}
        % Query
        \item \textbf{Query}
        \begin{itemize}
            \item Joao made a query in the database.
            \item A query finds information quickly.
        \end{itemize}
        % Quantum Computing
        \item \textbf{Quantum Computing}
        \begin{itemize}
            \item Quantum computing is very advanced.
            \item Quantum computers can solve hard problems.
        \end{itemize}
        % Queue
        \item \textbf{Queue}
        \begin{itemize}
            \item The printer has a print queue.
            \item A queue organizes tasks in order.
        \end{itemize}
    \end{itemize}
    % Subseccion letra R
    \subsection{Letra R}
    \begin{itemize}
        % Router
        \item \textbf{Router}
        \begin{itemize}
            \item The router connects to the internet.
            \item Every home network needs a router..
        \end{itemize}
        % RAM (Random Access Memory)
        \item \textbf{RAM (Random Access Memory)}
        \begin{itemize}
            \item RAM makes the computer run faster.
            \item Adding more RAM helps with multitasking.
        \end{itemize}
        % Repository
        \item \textbf{Repository}
        \begin{itemize}
            \item The code is stored in a repository.
            \item A repository helps with version control.
        \end{itemize}
        % Runtime
        \item \textbf{Runtime}
        \begin{itemize}
            \item The program needs a runtime environment.
            \item Runtime errors happen during execution.
        \end{itemize}
    \end{itemize}
    % Continua con las secciones

\section{CONCLUSIONES}

\section{RECOMENDACIONES}



\clearpage
\begin{thebibliography}{X}
    \bibitem{Autor1} Complete reference here
    \bibitem{Autor2} Complete reference here
    \bibitem{Autor3} Complete reference here
\end{thebibliography}